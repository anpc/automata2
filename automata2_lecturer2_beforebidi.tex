\documentclass{article}
\usepackage[cp1255]{inputenc}
\usepackage{amsmath}
\usepackage[english,hebrew]{babel}
\usepackage{biditex}
\usepackage{graphics}
\usepackage{tikz}
\usetikzlibrary{calc}
\tikzset{
    pics/circle vertically split/.style 2 args = {
       code = {
         \node[inner sep=3pt,left] (-left) {#1};
         \node[inner sep=3pt,right] (-right) {#2};
         \path let
              \p1 = ($(-left.north west) - (-left.east)$),
              \p2 = ($(-right.west) - (-right.south east)$),
              \n1 = {max(veclen(\p1), veclen(\p2))*2}
           in node[minimum size=\n1, circle, draw] (-shape) at (0,0) {};
         \draw (-shape.north) -- (-shape.south);
       }
    }
}


%BIDION
\begin{document}
\title{אוטומטים 2 - הרצאה מס' 2}
\date{26/03/17}
\author{}
\maketitle
\large
\textbf{\underline{משפט נרוד:}}
תהי
$L \subseteq{\Sigma^*}$,
אזי L רגולרית אם"ם ל-$R_{L}$ יש מספר סופי של מחלקות שקילות.

\setlength\parskip{\baselineskip}

נמשיך בהגדרות רלוונטיות איתן התחלנו בהרצאה הקודמת:

\setlength\parskip{\baselineskip}

\textbf{הגדרה:}
תהי $L \subseteq \Sigma^*$ ויהי R יחס שקילות מעל $\Sigma^*X \Sigma^*$. נאמר ש-R \underline{מעדן} את L אם"ם: $(x,y) \in R \rightarrow x \in L \Leftrightarrow y \in L$.

\setlength\parskip{\baselineskip}

(\emph{\underline{הערה חשובה:}} היחס מוגדר על $\Sigma^*$ ולא רק על L)

\setlength\parskip{\baselineskip}

\textbf{הגדרה:}
יהי
$R \subseteq{\Sigma^*X \Sigma^*}$
יחס שקילות. נאמר ש-R
\underline{אינווריאנטי מימין} אם לכל
$(x,y)\in \Sigma^*$ :\\
$(x,y)\in R \Rightarrow \forall z, (xz,yz)\in R$

\setlength\parskip{\baselineskip}

\underline{דוגמה 1:}

\setlength\parskip{\baselineskip}
%BIDIOFF
\inputencoding{latin9}\selectlanguage{english}
\begin{tikzpicture}
\draw (2,-1) pic (A) {circle vertically split={$a,b,ba$}{$\Sigma^*-\{a,b\}$}};
\end{tikzpicture}
\inputencoding{cp1255}\selectlanguage{hebrew}
%BIDION

\setlength\parskip{\baselineskip}

יחס זה אינו אינווריאנטי מימין. למה?\\
דוגמה נגדית: $(a,b) \in R$, אבל עבור $z=a$ : $(az,bz) \notin R$

\setlength\parskip{\baselineskip}

\underline{דוגמה 2:}
$R=\{(x,y)\in \{a,b\}^*X \{a,b\}^* | \#_{a}(x) \equiv \#_{a}(y)  mod 3\}$\\
היחס הזה אכן אינווריאנטי מימין. נראה למה:\\
יהי
$(x,y) \in R$\\
אזי
$\#_{a}(x) \equiv \#_{a}(y)  mod 3$     \textbf{(*)}\\
יהי
$z \in \Sigma^*$ אזי:
%BIDIOFF
\begin{equation*}
	\left\{
	\begin{array}{rl}
	\#_{a}(xz) = \#_{a}(x) + \#_{a}(z) \\
	\#_{a}(yz) = \#_{a}(y) + \#_{a}(z)
	\end{array}\right.
\end{equation*}
%BIDION
ומתוך \textbf{(*)} נובע:
$\#_{a}(xz) \equiv \#_{a}(yz)  mod 3  \Rightarrow (xz,yz) \in R$

\setlength\parskip{\baselineskip}

\textbf{\emph{טענה 1:}} יהי A DFA עם א"ב $\Sigma$.
אזי היחס $R_{a}$:\\
1) מעדן את $L(A)$\\
2) אינווריאנטי מימין\\
(\emph{\underline{הערה חשובה:}} בשיעור הקודם הראינו כי $R_{a}$ הוא יחס שקילות)

\setlength\parskip{\baselineskip}

\textbf{\emph{הוכחה:}}\\
1) יהיו $(x,y) \in R_{a}$ אזי:\\
$\hat{\delta}(q_{0},x) = \hat{\delta}(q_{0},y) = q \in Q$. לפיכך ייתכנו שני מקרים:\\
א. אם $q \in F$ אז $x,y \in L$ (ואז כמובן $x \in L \Leftrightarrow y \in L$ מתקיים).\\
ב. אם $q \notin F$ אז $x,y \notin L$ (וגם אז $x \in L \Leftrightarrow y \in L$ מתקיים. הפעם כ-False).

\setlength\parskip{\baselineskip}

2) יהיו $(x,y) \in {R_{a}}$ \textbf{(*)}. נראה שלכל $z \in \Sigma^*$ מתקיים $(xz,yz) \in R_{a}$:\\
$\hat{\delta}(q_{0},xz) = \hat{\delta}(\hat{\delta}(q_{0},x),z) \underset{by \textbf{(*)}}{=} \hat{\delta}(\hat{\delta}(q_{0},y),z) = \hat{\delta}(q_{0},yz)$\\
מ.ש.ל

\setlength\parskip{\baselineskip}

\textbf{\emph{טענה 2:}} תהי $L \subseteq \Sigma^*$ שפה. אז $R_{L}$ הוא:\\
1) יחס שקילות (הוכחנו בשיעור שעבר)\\
2) מעדן את L\\
3) אינווריאנטי מימין

\setlength\parskip{\baselineskip}

\textbf{\emph{הוכחה:}}\\
2) יהיו $(x,y) \in R_{L}$, אזי לכל z : $xz \in L \Leftrightarrow yz \in L$ (מהגדרת $R_{L}$).\\
נציב $z=\varepsilon$ ונקבל $x \in L \Leftrightarrow y \in L$ כנדרש (הגדרת "מעדן").\\
מ.ש.ל

\setlength\parskip{\baselineskip}

3) נניח $(x,y) \in R_{L}$ ונראה שלכל $z \in \Sigma^*$ מתקיים $(xz,yz) \in R_{L}$.\\
נקבע $z \in \Sigma^*, (x,y) \in R_{L}$ מסויימים. \underline{צ"ל:} לכל $w \in \Sigma^*$ מתקיים: $\big((xz)w,(yz),w\big) \in R_{L}$.\\
כייון ש $(x,y) \in R_{L}$ מתקיים: $x\underset{z'}{(zw)} \in L \Leftrightarrow y\underset{z'}{(zw)} \in L$.

\setlength\parskip{\baselineskip}

מאסוציאטיביות של שרשור מחרוזות, נוכל להזיז את הסוגריים ולקבל:\\
$\underset{(xz)w}{x(zw)} \in L \Leftrightarrow \underset{(yz)w}{y(zw)} \in L$\\
מ.ש.ל

\setlength\parskip{\baselineskip}

\textbf{\underline{משפט האפיון של $R_{L}$:}}
תהי  $L \subseteq \Sigma^*$, אזי אם $R \subseteq{\Sigma^*X \Sigma^*}$ הוא יחס שקילות המעדן את L ואינווריאנטי מימין, אזי R מעדן את $R_{L}$.\\
במילים אחרות: $R_{L}$ הינה החלוקה הכי גסה האפשרית של $\Sigma^*$ המעדנת את L ומתאימה ליחס שקילות אינווריאנטי מימין.

\setlength\parskip{\baselineskip}

\textbf{\underline{הוכחה:}}\\
יהיו $(x,y) \in R$, נרצה להראות שגם $(x,y) \in R_{L}$.\\
%BIDIOFF
\begin{equation*}
	(x,y) \in R \underset{right-invariant}{\Longrightarrow} \forall z,(xz,yz) \in R \underset{R-delights-L}{\Longrightarrow}
	\forall z,(xz \in L \Leftrightarrow yz \in L)\\
\end{equation*}
%BIDION
ומהגדרת $R_{L}$: $(xz,yz) \in R_{L}$.\\
מ.ש.ל

\setlength\parskip{\baselineskip}

כעת נוכיח את משפט נרוד.

\setlength\parskip{\baselineskip}

\textbf{\underline{הוכחת משפט נרוד:}}\\
\underline{כיוון $\Leftarrow$:} תהי $L \subseteq \Sigma^*$ שפה רגולרית. נראה כעת שמס' מחלקות השקילות של $R_{L}$ הוא סופי.\\
נסמן את מס' מחלקות השקילות של $R_{L}$ כך: $index(R_{L})$.\\
כיוון ש-L רגולרית, קיים DFA A שמקבל אותה. מטענה 1 $R_{A}$ מעדן ואינווריאנטי מימין. כמו כן: $index(R_{A}) \leq |Q|$.
זה מתקיים כמובן מכיוון שמחלקות השקילות של $R_{A}$ היא קבוצת ה-$L_{q}$ים הלא ריקים של A.\\

\setlength\parskip{\baselineskip}

ממשפט האפיון של $R_{L}$ מתקיים:\\
$index(R_{A}) \underset{R_{A}-delights-R_{L}}{\geq}index(R_{L})$.\\
לפיכך, $R_{L}$ בהכרח סופי.

\setlength\parskip{\baselineskip}

את הכיוון השני נראה בהרצאה הבאה.

\end{document}
%BIDIOFF	